\documentclass[11pt,titlepage]{article}
\usepackage{fullpage}
\usepackage{amsmath}
\usepackage{amssymb}
\usepackage{color}
\usepackage{graphicx}
\graphicspath{ {images/} }
\usepackage{tikz}
\usetikzlibrary{shapes,arrows,positioning,calc}
\usepackage{float}
\restylefloat{table}
\usepackage{array}
\tikzset{
    block/.style = {draw, fill=white, rectangle, minimum height=3em, minimum width=3em},
    sum/.style = {draw, fill=white, circle, node distance=1cm},
    input/.style = {draw=none},
    output/.style = {draw=none},
    coord/.style = {coordinate}
}

\author{Rane Brown \\ Kate Schneider}
\title{ECEN 4638: Lab X.1PI}
\date{\today}

\begin{document}
\maketitle
\tableofcontents
\listoffigures
\newpage

\section{Description}
    This lab will further explore the Torsional Disc System. The system setup will be similiar to what was used in labX.1P; only the bottom disc of the TDS will used and the four weights will be set at a radius of 6.5cm.

\section{System Model}
    \subsection{Calculated Parameters}

    \subsection{Transfer Functions}

\section{Matlab Analysis}
    \subsection{Time Domain}

    \subsection{Frequency Domain}

\section{Experimental Analysis}
    \subsection{Time Domain}

    \subsection{Frequency Domain}

\section{PI Controller Design}

\end{document}